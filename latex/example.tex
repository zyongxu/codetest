\documentclass[12pt,a4paper]{article}
\usepackage{amsmath}
%\usepackage[english,greek]{babel}
%\usepackage[iso-8859-7]{inputenc}
%-- end of preamble (the block between \documentclass and \begin{document})
\begin{document} % start of the "document" environment
\tableofcontents % this expands to an index of \sections

\section{Basics}
\subsection{Basic latex source file structure}
\begin{verbatim}
\documentclass{} % always as the first line in a latex source
[preamble]
\begin{document}
contents
\end{document}
\end{verbatim}

\subsection{Organize Contents}
A new document might need to be compiled multiple times to get a correct table of contents.
available sectioning commands:
\begin{verbatim}
\section{...}
\subsection{...}
\subsubsection{...}
\paragraph{...}
\subparagraph{...}
\end{verbatim}

\subsection{Environment}
\begin{tabular}{|c|}
    \hline
    \verb+\begin{environment} text \end{environment}+ \\
    \hline
\end{tabular}

Where environment is the name of the environment. Environments can be
nested within each other as long as the correct nesting order is maintained.
\begin{tabular}{|c|}
    \hline
    \verb+\begin{aaa}...\begin{bbb}...\end{bbb}...\end{aaa}+ \\
    \hline
\end{tabular}

\section[Dev Env]{Development Environment}
I'm using \underline{vim} + \emph{vimtex} to edit latex

\begin{verbatim}
NOTE: in my vim config <localleader> is "\"
\end{verbatim}

some useful shortcuts: \\ % line break (same as \newline)
\begin{tabular}{|l|l|c|}
\hline
key                     &Command                                             & MODE \\
\hline
\verb|<localleader>ll|  &vimtex-compile-toggle                               &n \\
\hline
\verb|<localleader>lo|  &vimtex-compile-output                               &n \\
\hline
\verb|<localleader>le|  &vimtex-errors                                       &n \\
\hline
\verb|]]|               &vimtex-delim-close (close the current environment)  &i \\
\hline
\end{tabular}

\section{Lists}
\begin{enumerate}
    \item ordered list item 1 \{enumerate\}
    \item ordered list item 2:
        \begin{itemize}
            \item unordered list item \{itemize\}
        \end{itemize}
    \item description list item 3:
        \begin{description}
            \item[desc1] item \{description\}
            \item[desc2] item
        \end{description}
    \item[-] item start with dash item 4
\end{enumerate}

\section{Equations}
% Example 1
\begin{quote}
\ldots when Einstein introduced his formula
\end{quote}
\begin{equation}
e = m \cdot c^2 \; ,
\end{equation}
``which is at the `same' time the most widely known
and the least well understood physical formula.''

% Example 2
\begin{equation}
\sum_{k=1}^{n} I_k = 0 \; .
\end{equation}

\section{Math Equations}
Package: amsmath (ams == American Mathematical Society) \\
\subsection{text style}
text style means an equation located in-line within a paragraph between two `\$'.

\subsection{display style}
display style means an equation breaks a paragraph between \verb+\begin{equation}+ and 
\verb+\end{equation}+

\subsection{Example}
This is text style:
$\lim_{n \to \infty}
\sum_{k=1}^n \frac{1}{k^2} = \frac{\pi^2}{6}$.\\
This is display style:
\begin{equation}
    \lim_{n \to \infty} 
    \sum_{k=1}^n \frac{1}{k^2} 
    = \frac{\pi^2}{6} \label{aRefLabel} % create label to refer to
\end{equation}
and this is a equation ref \eqref{aRefLabel}

\subsection{Greeks}
\begin{tabular}{|c|c|}
    \hline
    Command & Letter \\
    \hline
    \verb+$\alpha$+ & $\alpha$ \\
    \hline
    \verb+$A$+      & $A$ (uppercase alpha) \\
    \hline
    \verb+$\sigma$+ & $\sigma$ \\
    \hline
    \verb+$\Sigma$+ & $\Sigma$ \\
    \hline
\end{tabular}

\subsection{misc}
$p^3_{ij} \qquad
m_\text{Knuth}\qquad
\sum_{k=1}^3 k \\[5pt]
a^x+y \neq a^{x+y}\qquad
e^{x^2} \neq {e^x}^2$
$\sqrt{x} \Leftrightarrow x^{1/2}
\quad \sqrt[3]{2}
\quad \sqrt{x^{2} + \sqrt{y}}
\quad \surd[x^2 + y^2]$

\end{document} % end of the "document" environment
